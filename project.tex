\documentclass{article}
\title{Gruppenarbeit: Auflistung von Projekten Gruppe: Phis n' Chips}
\author{Anna Kaiser, Sina Wissmann, Gwendolyn Crestan, Jepser Steuer, Maurice Mick, Simon Müller}
\date{\today}

\begin{document}
\maketitle 
\section{Einleitung}
In diesem Dokument werden verschiedene Projekte aufgelistet, die im Rahmen der Gruppenarbeit analysiert wurden.


\newpage

\section{Projekte}
\begin{itemize}
    \item Projekt 1: The Human Brain Project (HBP) / EBRAINS
    \item Projekt 2: HOME Gebäude in Mannheim
    \item Projekt 3: Microsoft TALE
\end{itemize}


\section{Projekt 1: The Human Brain Project (HBP) / EBRAINS}
\begin{itemize}
    \item \textbf{Projektbeschreibung:} 
    Das "The Human Brain Porject" war ein großes europäisches Forschungsprojekt, das darauf abzielte, 
    das menschliche Gehirn zu simmulieren. Diese Simulation sollte mittels Supercomputing ermöglicht 
    werden und eine flächenübergreifende Infrastruktur zur Forschung aufgebaut werden. 
    So könne zum einen die Hirnforaschung, als auch die Entwicklung neuer Computerarchitekturen 
    vorangetrieben werden.

    \item \textbf{Ziele:}
    \begin{itemize}
        \item Das Menschliche Gehirn sollte am Computer so genau wie möglich nachgebaut und
        \item Es sollt ein genaueres Verständnis der Funktionsweise des Gehirns erlangt 
        werden, damit Krankheiten wie Altzheimer oder Demenz besser verstanden werden können.
        \item Bereitstellung einer Cloudplattform (EBRAINS) für die kollaborative 
        Hirnforschung.
        \item Vorrantreiben der Forschung auf dem Gebiet der neuartigen Computerarchitekturen.
    \end{itemize}

    \item \textbf{Begrenzungen:}
    \begin{itemize}
        \item Zeitlich: Das Projekt war von Anfang an als Langzeitprojekt mit einer Laufzeit von 10 Jahren ausgelegt.
        Es lief von 2013 bis 2023. Die daraus hervorgegangene Infrastruktur ENBRAINS wird weiter betrieben, jedoch ist das ursprüngliche Projekt abgeschlossen.
        \item Finanziell: Da das Projekt ein Europaisches Flagshipprojekt war, wurde es mit einem Budget von 600 Millionen Euro gefördert.
        \item Personell: An diesem Projekt arbeiteten hunderte Wissenschaflter aus über 150 Institutionen daran. Dies bedeutete einen enormen Koordinationsaufwand.
    \end{itemize}

    \item \textbf{Organisationsform:}
    Da das Projekt eine enorme Größe, als Verbund von Universitäten und Forschungseinrichtungen aus 19 verschiedenen Ländern, hatte, gab es eine zentrale 
    Koordination in der Schweiz. Die Projektdurchführung erfolgte jedoch dezentral an den verschiedenen Partnerinstitutionen.
\end{itemize}

\end{document}