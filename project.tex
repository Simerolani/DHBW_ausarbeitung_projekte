\documentclass{article}
\usepackage[utf8]{inputenc}
\usepackage[T1]{fontenc}
\usepackage{graphicx} % Notwendig für das Einfügen von Bildern
\usepackage[firstpage=true]{background} % Lädt das Hintergrund-Paket für die erste Seite
\usepackage{xcolor} % Für Farben
\usepackage{titlesec} % Für Überschriften-Design

\definecolor{DesignRed}{RGB}{221, 0, 25}
\titleformat{\section}
  {\normalfont\Large\bfseries\color{DesignRed}} % Format des Titels
  {\thesection} % Die Nummer (z.B. "1.")
  {1em} % Abstand zwischen Nummer und Titel
  {} % Code vor dem Titel (lassen wir leer)

\backgroundsetup{
  contents = {\includegraphics[ % Der Inhalt ist unser Bild
    width=\paperwidth,   % Das Bild soll so breit wie das Papier sein
    height=\paperheight  % Das Bild soll so hoch wie das Papier sein
  ]{Bilder/Hintergrund_dreiecke.png}}, % Name deiner Bilddatei
  scale = 1,      % Keine zusätzliche Skalierung
  opacity = 1,    % Keine Transparenz
  angle = 0,      % Nicht drehen
  position = current page % Positioniert das Bild auf der ganzen Seite
}

\title{
    \vspace*{5cm}
    Gruppenarbeit: Auflistung von Projekten \\ Gruppe: Phis n' Chips}
\author{Anna Kaiser, Sina Wissmann, Gwendolyn Crestan, \\ Jepser Steuer, Maurice Mick, Simon Müller}
\date{\today}

\begin{document}
\maketitle
\clearpage % Stellt sicher, dass der Hintergrund nur auf der ersten Seite ist
\section{Einleitung}
In diesem Dokument werden verschiedene Projekte aufgelistet, die im Rahmen der Gruppenarbeit analysiert wurden.

\section{Projekte}
\begin{itemize}
    \item Projekt 1: The Human Brain Project (HBP) / EBRAINS
    \item Projekt 2: Das FRANKLIN-Projekt in Mannheim
    \item Projekt 3: Microsoft TALE
\end{itemize}

\clearpage

%Projekt 1
\section{Projekt 1: The Human Brain Project (HBP) / EBRAINS}
\begin{itemize}
    \item \textbf{Projektbeschreibung:} 
    Das "The Human Brain Project" (HBP) war ein großes europäisches Forschungsprojekt, das darauf abzielte, 
    das menschliche Gehirn zu simulieren. Diese Simulation sollte mittels Supercomputing ermöglicht 
    werden und eine flächenübergreifende Infrastruktur zur Forschung aufgebaut werden. 
    So könne zum einen die Hirnforschung, als auch die Entwicklung neuer Computerarchitekturen 
    vorangetrieben werden. Das Projekt mündete in die EBRAINS-Plattform.

    \item \textbf{Ziele:}
    \begin{itemize}
        \item Es sollte ein genaueres Verständnis der Funktionsweise des Gehirns erlangt 
        werden, damit Krankheiten wie Altzheimer oder Demenz besser verstanden werden können.
        \item Bereitstellung einer Cloudplattform (EBRAINS) für die kollaborative 
        Hirnforschung.
        \item Simulation des menschlichen Gehirns (wurde nicht vollständig erreicht).
        \item Vorrantreiben der Forschung auf dem Gebiet der neuartigen Computerarchitekturen.
    \end{itemize}

    \item \textbf{Begrenzungen:}
    \begin{itemize}
        \item \textbf{Zeitlich:} Das Projekt war von Anfang an als Langzeitprojekt mit einer Laufzeit von 10 Jahren ausgelegt (2013–2023).
        \item \textbf{Finanziell:} Da das Projekt ein Europäisches Flaggschiffprojekt war, wurde es mit einem Budget von rund 600 Millionen Euro gefördert.
        \item \textbf{Personell:} An diesem Projekt arbeiteten hunderte Wissenschaftler aus über 150 Institutionen. Dies bedeutete einen enormen Koordinationsaufwand.
    \end{itemize}

    \item \textbf{Abgrenzung:}
    \begin{itemize}
        \item \textbf{Neuartig:} Das HBP war neuartig in seinem interdisziplinären Ansatz, der Neurowissenschaften, 
        Supercomputing und Medizin auf einer einzigen Plattform (EBRAINS) verband. 
        Es war der erste Versuch, die Hirnforschung in diesem massiven, datengesteuerten Maßstab zu betreiben.
        \item \textbf{Einmalig:} Die schiere Größe, das Budget und die Koordination von 150 Institutionen aus 19 Ländern machten das Projekt in seiner Form einmalig.
        \item \textbf{Komplex:} Die Komplexität lag nicht nur in der wissenschaftlichen Herausforderung (dem Gehirn selbst), 
        sondern auch im Management und der Koordination Tausender von Datenquellen und unterschiedlicher Forschungsansätze.
        \item \textbf{Risikoreich:} Das Projekt war von Anfang an hoch riskant. Das Risiko des Scheiterns (insbesondere beim Ziel der vollständigen Simulation) war groß, 
        was zu öffentlicher Kritik und einer strategischen Neuausrichtung während der Laufzeit führte.
    \end{itemize}

    \item \textbf{Organisationsform:}
    Da das Projekt eine enorme Größe (als Verbund von Universitäten und Forschungseinrichtungen aus 19 verschiedenen Ländern) hatte, gab es eine zentrale 
    Koordination in der Schweiz. Die Projektdurchführung erfolgte jedoch dezentral an den verschiedenen Partnerinstitutionen.

    \item \textbf{Abschließende SMART Bewertung:}
    Die Ziele des HBP waren Spezifisch (Schaffung von EBRAINS, Erforschung von Krankheiten), 
    Messbar (anhand von Publikationen, Nutzerzahlen der Plattform) und Attraktiv (EU-Flaggschiff). 
    Das Projekt war klar terminiert (10 Jahre). 
    Die Realitätsnähe war jedoch umstritten, da das Hauptziel einer vollständigen Gehirnsimulation als unrealistisch galt, 
    während die Teilziele (Aufbau der Plattform EBRAINS, Fortschritte in Teilbereichen) realistisch erreicht wurden.

\end{itemize}
\clearpage



%Projekt 2

\section{Projekt 2: HOME Gebäude in Mannheim}
\begin{itemize}
    \item \textbf{Projektbeschreibung:} 
    Das FRANKLIN-Projekt in Mannheim ist ein innovatives Bauvorhaben, das darauf abzielt,
    nachhaltige und energieeffiziente Wohn- und Arbeitsräume zu schaffen.

    \item \textbf{Ziele:}
    Das Ziel des Projekts ist, auf dem rund 144 ha großen ehemaligen US-Militärgelände ein neues, 
    eigenständiges Stadtquartier zu schaffen, das etwa 4.700 Wohneinheiten für rund 10.500 Menschen umfasst. 
    Gleichzeitig sollen rund 35 \% der Fläche als Grün- und Freiraum ausgelegt, 
    verschiedene Wohnformen angeboten und ein hoher Anspruch an Energie- und Klimaschutz realisiert werden.

    \item \textbf{Begrenzungen:}
    Das Vorhaben ist zeitlich auf eine mehrphasige Umsetzung angelegt – das gesamte Quartier soll schrittweise entwickelt werden, 
    ein fixer Endtermin ist nicht klar definiert. Finanziell bestehen Limitierungen durch ein u. a. durch Investoren gegebenes Budget, 
    aber auch Vorgaben wie z. B. ein Drittel preisgünstiger Wohnungen sowie durch hohe Infrastruktur- und Erschließungskosten. 
    Personell wird das Projekt durch eine zentrale Entwicklungs­gesellschaft (MWSP) gesteuert, mit begrenzten Kapazitäten und zahlreichen Beteiligten, 
    was Koordination und Kommunikation erfordert.

    \item \textbf{Organisationsform:}
    Das Projekt wird von der Stadt Mannheim über die Entwicklungsgesellschaft MWSP (Mannheimer Wohnungsbau- und Stadtentwicklungsgesellschaft mbH) gesteuert. 
    Ein Masterplan wurde extern erstellt, Wettbewerbe und Bürgerbeteiligung genutzt. 
    Die Umsetzung erfolgt in Teilquartieren mit verschiedenen Bauträgern und Investoren. 
    Ein „Beraterkreis“ soll Qualität von Architektur und Städtebau sichern.

    \item \textbf{Wie grenzt(e) sich das Projekt ab?:}
    Das Projekt ist neuartig, da eine militärische Konversionsfläche in ein modernes, 
    durchmischtes Stadtquartier mit Nachhaltigkeitsfokus umgewandelt wird. 
    Es ist einmalig, da der Standort spezifisch ist und nicht einfach kopierbar. 
    Es ist komplex, aufgrund der Größe, Vielfalt von Nutzungen, Bestandsgebäude­integration und Infrastruktur-anforderungen. 
    Es ist zudem risikoreich: Durch bekannte Problemfaktoren bei Großbauprojekten kam es auch hier zu finanziellen und zeitlichen Rückschlägen. 
    Aktuelle Kritik klagt z. B., dass bei Bewohnern mangelnde Infrastruktur, Parkraum-Probleme und unklare Zeitpläne zu Frust führen. 
    Auch der Zeitplan für die Vollendung markanter Hochpunkte (Hochhäuser, die die Buchstaben „HOME“ formen) verzögern sich, 
    da die Finanzierung des „M“-Hauses durch Absprung des Investors lange Zeit unklar war. Durch Bau der restlichen Häuser prangt aktuell „HOE“ in Franklin.

    \item \textbf{Abschließende SMART Bewertung:}
    Die Zielvorgaben sind spezifisch (Fläche, Wohneinheiten, Freiflächenanteil), 
    messbar (Zahlen genannt), relevant (Wohnraumbedarf, Nachhaltigkeit). Auch ist es grundlegend realistisch und terminiert, 
    da der Entwicklungsprozess eingerichtet ist. Dennoch zeigen gerade die bereits entstandenen Probleme, 
    dass das Projekt hinsichtlich „R“ und „T“ herausfordernder ist. 
  \end{itemize}
\clearpage

%Projekt 3
\section{Projekt 3: Microsoft TAY}
\begin{itemize}
  \item \textbf{Projektbeschreibung:} 
  Microsoft TAY war ein experimenteller Chatbot, der entwickelt wurde, um
  natürliche Gespräche mit Menschen zu führen und dabei von ihnen zu lernen. Wurde jedoch nach kurzer Zeit aufgrund von Missbrauch durch Nutzer offline genommen.
  
  \item \textbf{Ziele (SMART-Formulierung):}
  \begin{itemize}
    \item \textbf{Spezifisch:} Das Projekt Microsoft Tay hatte das klare Ziel, einen lernfähigen Chatbot zu entwickeln, der auf sozialen Netzwerken, 
    insbesondere Twitter, mit jungen Erwachsenen (18–24 Jahre) kommuniziert. 
    Tay sollte auf natürliche Weise interagieren und den Sprachstil der Zielgruppe übernehmen, um authentische, menschlich wirkende Dialoge zu erzeugen.
    \item \textbf{Messbar:} Der Erfolg des Projekts sollte anhand der Kommunikationshäufigkeit, der Gesprächsdauer und
    der Interaktionsqualität mit Nutzerinnen und Nutzern gemessen werden. Ziel war es, eine hohe 
  Engagement-Rate zu erreichen und positive Rückmeldungen zur Natürlichkeit der Gespräche zu erhalten.
    \item \textbf{Attraktiv:} Das Projekt war technologisch und gesellschaftlich attraktiv, da es einen innovativen Schritt im 
    Bereich der künstlichen Intelligenz darstellte. Microsoft wollte zeigen, dass KI in der Lage ist, 
    selbstständig Sprache zu lernen und menschenähnlich zu agieren; ein Meilenstein für zukünftige KI-Anwendungen im Kundenkontakt und Social Media.
    \item \textbf{Realisitisch:} Die Umsetzung war realistisch, da Microsoft bereits über umfangreiche Ressourcen im Bereich 
    des maschinellen Lernens und der Sprachverarbeitung verfügte. Außerdem hatte das 
    Unternehmen in China bereits einen ähnlichen Chatbot namens XiaoIce herausgebracht, der ein 
    großer Erfolg war. Dennoch stellte das Projekt eine erhebliche Herausforderung dar, da es sich 
    um einen der ersten Chatbots handelte, der direkt aus Nutzerinteraktionen lernte, was strenge Filtermechanismen erforderlich machte.
    \item \textbf{Terminiert:} Das Projekt war von Anfang an als zeitlich begrenztes Experiment konzipiert. Geplant war eine 
    mehrwöchige Testphase mit fortlaufender Analyse. Anschließend sollte Tay auch für andere 
    demografische Gruppen angepasst werden. Tatsächlich lief der Test jedoch nur vom 23. bis 24. März 2016, da Tay 
    bereits nach rund 16 Stunden wegen unangemessenen Verhaltens deaktiviert werden musste.
  \end{itemize}

  \item \textbf{Begrenzungen:}
  \begin{itemize}
    \item \textbf{Zeitlich:} Die ursprünglich geplante Laufzeit des Projekts erstreckte sich über mehrere Wochen. 
    Aufgrund fehlender Schutzmechanismen für die Lernprozesse wurde Tay jedoch bereits nach 16 Stundenoffline genommen.
    \item \textbf{Finanziell:} Das Projekt wurde vollständig über das Forschungsbudget von Microsoft Research finanziert. 
    Da es sich um ein internes Experiment ohne kommerziellen Hintergrund handelte, waren die finanziellen Mittel für Überwachung, Sicherheit und Nachsteuerung begrenzt.
    \item \textbf{Personell:} Das Projektteam setzte sich aus Microsofts Technology and Research-Team sowie dem Bing- Team zusammen und 
    umfasste Forscher und Ingenieure aus den Bereichen maschinelles Lernen und große Sprachmodelle (LLMs) sowie weitere Wissenschaftler. 
    Eine dauerhafte ethische Kontrolle war nicht Teil der Struktur, was sich im Verlauf des Projekts als entscheidender Schwachpunkt erwies.
  \end{itemize}

  \item \textbf{Abgrenzung:}
  \begin{itemize}
    \item \textbf{Neuartig:} Zwar gab es bereits zuvor Chatbots, die mit Nutzern auf sozialen Plattformen interagierten, 
    etwa XiaoIce in China, doch Tay war der erste Chatbot, der frei auf einer öffentlichen Social-Media-Plattform interagierte 
    und dabei direkt aus menschlichen Gesprächen lernte. Frühere Chatbots basierten auf festen Antwortmustern, 
    während Tay dynamisch auf neue Situationen reagierensollte.
    \item \textbf{Einmalig:} Das Projekt war in seiner Form einzigartig. Microsoft hatte zuvor keine vergleichbaren Experimente in einem unkontrollierten, öffentlichen Umfeld durchgeführt.
    \item \textbf{Komplex:}Das Projekt kombinierte verschiedene Fachbereiche: maschinelles Lernen, Sprachverarbeitung, 
    sowie Echtzeit-Interaktion. Diese Kombination machte Tay zu einem hochkomplexen System mit zahlreichen potenziellen Einflussfaktoren
    \item \textbf{Risikoreich:} Da Tay unkontrolliert von Nutzerinnen und Nutzern lernen konnte, bestand ein erhebliches Missbrauchsrisiko. 
    Bereits nach wenigen Stunden begannen Personen, den Chatbot gezielt mit rassistischen, beleidigenden und diskriminierenden Inhalten zu füttern, 
    woraufhin Tay selbst solche Aussagen verbreitete. Dies führte letztlich zum schnellen Abbruch des Projekts.
  \end{itemize}

  \item \textbf{Organisation:}
  \begin{itemize}
    \item \textbf{Projektorganisation:} Das Projekt wurde innerhalb von Microsoft Research in Zusammenarbeit mit dem Bing-Team durchgeführt. 
    Es handelte sich um eine eigenständige Projektorganisation, die unabhängig von anderen Microsoft-Abteilungen agierte. 
    Das Team setzte sich aus Spezialistinnen und Spezialisten für Künstliche Intelligenz, maschinelles Lernen, Sprachwissenschaft und Softwareentwicklung zusammen.
    \item \textbf{Fehlende Kontrollmechanismen:} Obwohl eine klare Projektleitung vorhanden war, fehlten Mechanismen zur ethischen oder sicherheitstechnischen Überwachung. 
    Dies führte dazu, dass problematische Lernmuster nicht rechtzeitig erkannt oder korrigiert werden konnten.
    \item \textbf{Organisatorische Konsequenzen:} Nach dem Scheitern des Projekts zog Microsoft wichtige Lehren: Im Jahr 2019 wurde das „Office of Responsible AI“ gegründet, 
    und bei der Entwicklung neuer KI-Chatbots wie Zo wurde darauf geachtet, die notwendigen Sicherheitsmechanismen zu integrieren, um unethische und anstößige Äußerungen zu verhindern. 
    Diese organisatorische Weiterentwicklung gilt heute als ein bedeutender Schritt hin zu einer verantwortungsbewussten KI-Entwicklung innerhalb des Unternehmens
  \end{itemize}
  
  
  
  \end{itemize}
\end{document}