\documentclass{article}
\usepackage[utf8]{inputenc}
\usepackage[T1]{fontenc}
\usepackage{graphicx} % Notwendig für das Einfügen von Bildern
\usepackage[firstpage=true]{background} % Lädt das Hintergrund-Paket für die erste Seite
\usepackage{xcolor} % Für Farben
\usepackage{titlesec} % Für Überschriften-Design

\definecolor{DesignRed}{RGB}{221, 0, 25}
\titleformat{\section}
  {\normalfont\Large\bfseries\color{DesignRed}} % Format des Titels
  {\thesection} % Die Nummer (z.B. "1.")
  {1em} % Abstand zwischen Nummer und Titel
  {} % Code vor dem Titel (lassen wir leer)

\backgroundsetup{
  contents = {\includegraphics[ % Der Inhalt ist unser Bild
    width=\paperwidth,   % Das Bild soll so breit wie das Papier sein
    height=\paperheight  % Das Bild soll so hoch wie das Papier sein
  ]{Bilder/Hintergrund_dreiecke.png}}, % Name deiner Bilddatei
  scale = 1,      % Keine zusätzliche Skalierung
  opacity = 1,    % Keine Transparenz
  angle = 0,      % Nicht drehen
  position = current page % Positioniert das Bild auf der ganzen Seite
}

\title{
    \vspace*{5cm}
    Gruppenarbeit: Auflistung von Projekten \\ Gruppe: Phis n' Chips}
\author{Anna Kaiser, Sina Wissmann, Gwendolyn Crestan, \\ Jepser Steuer, Maurice Mick, Simon Müller}
\date{\today}

\begin{document}
\maketitle
\clearpage % Stellt sicher, dass der Hintergrund nur auf der ersten Seite ist
\section{Einleitung}
In diesem Dokument werden verschiedene Projekte aufgelistet, die im Rahmen der Gruppenarbeit analysiert wurden.

\section{Projekte}
\begin{itemize}
    \item Projekt 1: The Human Brain Project (HBP) / EBRAINS
    \item Projekt 2: HOME Gebäude in Mannheim
    \item Projekt 3: Microsoft TALE
\end{itemize}

\clearpage
\section{Projekt 1: The Human Brain Project (HBP) / EBRAINS}
\begin{itemize}
    \item \textbf{Projektbeschreibung:} 
    Das "The Human Brain Porject" war ein großes europäisches Forschungsprojekt, das darauf abzielte, 
    das menschliche Gehirn zu simmulieren. Diese Simulation sollte mittels Supercomputing ermöglicht 
    werden und eine flächenübergreifende Infrastruktur zur Forschung aufgebaut werden. 
    So könne zum einen die Hirnforaschung, als auch die Entwicklung neuer Computerarchitekturen 
    vorangetrieben werden.

    \item \textbf{Ziele:}
    \begin{itemize}
        \item Es sollte ein genaueres Verständnis der Funktionsweise des Gehirns erlangt 
        werden, damit Krankheiten wie Altzheimer oder Demenz besser verstanden werden können.
        \item Bereitstellung einer Cloudplattform (EBRAINS) für die kollaborative 
        Hirnforschung.
        \item Vorrantreiben der Forschung auf dem Gebiet der neuartigen Computerarchitekturen.
    \end{itemize}

    \item \textbf{Begrenzungen:}
    \begin{itemize}
        \item Zeitlich: Das Projekt war von Anfang an als Langzeitprojekt mit einer Laufzeit von 10 Jahren ausgelegt.
        Es lief von 2013 bis 2023. Die daraus hervorgegangene Infrastruktur ENBRAINS wird weiter betrieben, jedoch ist das ursprüngliche Projekt abgeschlossen.
        \item Finanziell: Da das Projekt ein Europaisches Flagshipprojekt war, wurde es mit einem Budget von 600 Millionen Euro gefördert.
        \item Personell: An diesem Projekt arbeiteten hunderte Wissenschaflter aus über 150 Institutionen daran. Dies bedeutete einen enormen Koordinationsaufwand.
    \end{itemize}

    \item \textbf{Organisationsform:}
    Da das Projekt eine enorme Größe, als Verbund von Universitäten und Forschungseinrichtungen aus 19 verschiedenen Ländern, hatte, gab es eine zentrale 
    Koordination in der Schweiz. Die Projektdurchführung erfolgte jedoch dezentral an den verschiedenen Partnerinstitutionen.
\end{itemize}

\end{document}